%----------------------------------------------------------------------------------------
%	PACKAGES AND OTHER DOCUMENT CONFIGURATIONS
%----------------------------------------------------------------------------------------

\documentclass[11pt,fleqn]{book} % Default font size and left-justified equations

%----------------------------------------------------------------------------------------

\input{structure} % Insert the commands.tex file which contains the majority of the structure behind the template

\begin{document}

\pagenumbering{roman}
%----------------------------------------------------------------------------------------
%	TITLE PAGE
%----------------------------------------------------------------------------------------

\begingroup
\thispagestyle{empty}
\begin{tikzpicture}[remember picture,overlay]
\coordinate [below=18cm] (midpoint) at (current page.north);
\node at (current page.north west)
{\begin{tikzpicture}[remember picture,overlay]
\node[anchor=north west,inner sep=0pt] at (0,0) {\includegraphics[width=\paperwidth]{background}}; % Background image
\draw[anchor=north] (midpoint) node [text opacity=1,inner sep=1cm]{\Huge\centering\bfseries\sffamily\parbox[c][][t]{\paperwidth}{\centering CiViEl\\[20pt] % Book title
{\Large Cloud-based virtual laboratory}\\[20pt] % Subtitle
{\huge Patrizio Tufarolo}\\
{\small Progetto di esame - Corso di ``Gestione Progetti'' - Prof G. Gianini}\\
{\small A.A. 2015/2016 - 2016/2017 - Matr. 875041}


}}; % Author name
\end{tikzpicture}};
\end{tikzpicture}
\vfill
\endgroup
%----------------------------------------------------------------------------------------
%	COPYRIGHT PAGE
%----------------------------------------------------------------------------------------

\newpage
\noindent
\textbf{Versione:} 1.0\\
\textbf{Destinato a:} Analisti, sviluppatori web, sistemisti, UI-UX designer, sviluppatori del sistema in esame\\
\textbf{Metodologia:} I requisiti funzionali e non funzionali sono stati classificati sulla base delle seguenti priorità:\\
\begin{itemize}
    \item[\textbf{MUST}]: requisiti che il sistema deve soddisfare per fornire le funzionalità sotto esplicitate
    \item[\textbf{SHOULD}]: requisiti che il sistema dovrebbe avere per migliorarne l'accessibilità e l'accettabilità
    \item[\textbf{MAY}]: requisiti che migliorerebbero il sistema, da aggiungere nel caso in cui le tempistiche e il budget lo permettano
\end{itemize}
\textit{I diagrammi UML sono stati realizzati con il toolkit PlantUML \(http://plantuml.com/\), un componente che permette di creare diagrammi in modo dichiarativo}
\\
~\vfill
\thispagestyle{empty}

\noindent Copyleft 2017 Patrizio Tufarolo\\ % Copyright notice

\noindent Licensed under the Creative Commons Attribution-NonCommercial 3.0 Unported License (the ``License''). You may not use this file except in compliance with the License. You may obtain a copy of the License at \url{http://creativecommons.org/licenses/by-nc/3.0}. Unless required by applicable law or agreed to in writing, software distributed under the License is distributed on an \textsc{``as is'' basis, without warranties or conditions of any kind}, either express or implied. See the License for the specific language governing permissions and limitations under the License.\\ % License information

\noindent \textit{First printing, March 2017} % Printing/edition date

%----------------------------------------------------------------------------------------
%	TABLE OF CONTENTS
%----------------------------------------------------------------------------------------

%\usechapterimagefalse % If you don't want to include a chapter image, use this to toggle images off - it can be enabled later with \usechapterimagetrue
\usechapterimagetrue
\chapterimage{chapter_head_1.pdf} % Table of contents heading image

\pagestyle{empty} % No headers

\tableofcontents % Print the table of contents itself
\pagenumbering{arabic}
\cleardoublepage % Forces the first chapter to start on an odd page so it's on the right
\setcounter{page}{1}
\pagestyle{fancy} % Print headers again

\usechapterimagefalse
\subfile{capitoli/introduzione.tex}
\part{Fase di analisi}
\subfile{capitoli/catalogo-dei-requisiti.tex}
\subfile{capitoli/casi-d-uso.tex}
\subfile{capitoli/diagrammi-sequenza-analisi.tex}
\subfile{capitoli/domain-model.tex}
\part{Fase di design}
\subfile{capitoli/architettura-sistema.tex}
\subfile{capitoli/diagramma-deployment.tex}
\subfile{capitoli/modello-sistema.tex}
\subfile{capitoli/diagrammi-sequenza-design.tex}
\subfile{capitoli/diagramma-classi-design.tex}
\subfile{capitoli/diagrammi-stato-design.tex}
\part{Testing}
\subfile{capitoli/unit.tex}
\subfile{capitoli/subsystem.tex}
\subfile{capitoli/acceptance.tex}
\part{Pianificazione}
\subfile{capitoli/wbs.tex}
\subfile{capitoli/actnetwork.tex}
\subfile{capitoli/gantt.tex}



\cleardoublepage
\phantomsection
\setlength{\columnsep}{0.75cm}
\addcontentsline{toc}{chapter}{\textcolor{myblue}{Index}}
\printindex

%----------------------------------------------------------------------------------------

\end{document}
