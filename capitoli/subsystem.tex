\documentclass[../main.tex]{subfiles}
\begin{document}
\chapter{Casi di test di integrazione e di sistema}

Il prodotto descritto prevede l'integrazione con vari altri componenti. In particolare, come illustrato nei capitoli precedenti, le interazioni si svolgono con i seguenti moduli esterni
\begin{itemize}
    \item Database LDAP
    \item Container Engine
    \item Kernel Linux
\end{itemize}

\clearpage

\begin{tabularx}{\linewidth}{|c|X|X|X|c|}
    \toprule
    \textbf{Codice}      & \textbf{Classe}        & \textbf{Metodo}        & \textbf{Oracolo} & \textbf{Autom.} \\ 
    \midrule
    \endhead
    ST001       & LDAPServer    & checkCredentials         & Errore se non è possibile consultare il database LDAP con credenziali di autenticazione e cifratura valide & Si \\ \hline
    ST002       & ContainerEngine    & createContainer         & Errore se viene restituito un errore nella creazione di un container & Si \\ \hline
    ST003       & ContainerEngine    & removeContainer         & Errore se viene restituito un errore nella rimozione di un container & Si \\ \hline
    ST004       & ContainerEngine    & getContainerTTY         & Errore se viene restituito un errore nell'attachment alla TTY di un container & Si \\ \hline 
    ST005       & KernelLinux    & createNetNS         & Errore se il namespace non viene creato & Si \\ \hline
    ST006       & KernelLinux    & removeNetNS         & Errore se il namespace non viene rimosso & Si \\ \hline
    ST007       & KernelLinux    & createFS         & Errore se il filesystem non viene creato & Si \\ \hline
    ST008       & KernelLinux    & removeFS         & Errore se il filesystem non viene rimosso & Si \\ \hline
\end{tabularx}
\end{document}
