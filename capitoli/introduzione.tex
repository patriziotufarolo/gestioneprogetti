\documentclass[../main.tex]{subfiles}
\begin{document}
\chapter{Introduzione}
\section{Richiesta}
L'obiettivo del progetto è la realizzazione di una piattaforma di laboratorio informatico virtuale, acccessibile tramite tecnologie web, da utilizzare per fini didattici (esercitazioni pratiche nelle discipline che le prevedono) e per fini di ricerca.
Si richiede dunque una piattaforma multi utente e multi ambiente, che consenta all'utente di realizzare esperimenti in un ambiente simulato.
È richiesto in particolare lo sviluppo di un ambiente specifico per la realizzazione di un laboratorio di "Reti di Calcolatori", che
\begin{itemize}
    \item offra la possibilità agli utenti di disegnare topologie di rete coprendo i livelli I, II e III dello stack ISO/OSI
    \item consenta agli stessi di eseguire il \textit{deployment} della topologia disegnata
\end{itemize}

\paragraph{Disegno della topologia}
L'utente deve:
\begin{enumerate}
    \item Poter lavorare in un ambiente \textit{multi-canvas}, che consenta una migliore organizzazione del progetto
    \item Poter disegnare in ciascun \textit{canvas} il livello fisico della topologia, aggiungendo tramite un'interfaccia grafica \textit{drag-and-drop} delle icone rappresentanti i singoli dispositivi appartenenti alla topologia stessa.
    \item Poter collegare, tramite linee, le icone precedentemente disegnate. È prevista la possibilità di effettuare collegamenti tra dispositivi appartenenti a canvas diversi \textit{cross-canvas linking}.
    \item Poter inserire annotazioni grafiche nel canvas (forme geometriche, etichette).
\end{enumerate}
\paragraph{Deployment della topologia}
\begin{enumerate}
    \item Poter configurare tramite l'interfaccia grafica del prodotto i vari dispositivi e i link in fase di pre-deployment
    \item Eseguire il deployment della topologia di rete
    \item Applicare configurazioni sui link (limitazione della banda, aumento del bit-error-ratio ecc.)
    \item Poter accedere ai terminali di ciascuno dei dispositivi appartenenti il canvas in run-time, con la possibilità di installare e configurare software all'interno degli stessi

\end{enumerate}
\paragraph{Informazioni aggiuntive alla richiesta}
\begin{itemize}
    \item Il prodotto deve essere accessibile via web ed integrato con un sistema di autenticazione centralizzata in dotazione (LDAP)
    \item È definito il concetto di \textit{sessione di lavoro}, che non coincide con la sessione di login. Un utente deve poter effettuare login e logout più volte dalla piattaforma, ed accedere alle sessioni di lavoro a lui disponibile in qualunque momento.
    \item La sessione di lavoro è creata da un "supervisore di sessione", ovvero un utente con privilegi speciali.
    \item Deve essere fornita una modalità "supervisore di sessione", oltre a poter eseguire tutte le operazioni di cui sopra, ha la possibilità di interagire con le sessioni degli altri utenti, tramite l'implementazione di un'apposita modalità "actor".
    \item Le risorse create dall'utente devono rimanere disponibili all'interno della stessa sessione di lavoro, che non coincide con la sessione di login.
\end{itemize}

\section{Elicitazione}
Si intende offrire una piattaforma web-based che soddisfi i requisiti esplicitati nella richiesta e che possa essere installata in un ambiente distribuito al fine di poter garantire i requisiti impliciti di scalabilità richiesti dall'utilizzo intensivo multi-utente.

L'applicazione implementerà il concetto di \textit{multi-tenancy} al fine di garantire la separazione degli ambienti nell'utilizzo multi-utente.

L'approccio modulare che ne caratterizzerà lo sviluppo consentirà di espandere la stessa mediante la realizzazione di ambienti legati anche ad altre discipline.

Per quanto riguarda l'ambiente "Reti di calcolatori" richiesto, le funzionalità saranno implementate tramite l'adozione di tecnologie SDN\footnote{\textit {Software Defined Network}, approccio all'orchestrazione delle risorse di rete orientato al software.}

La parte operativa sarà realizzata mediante software open-source, facendo largo uso delle caratteristiche del kernel Linux\footnote{http://www.linux.org}.

La piattaforma disporra di una dashboard grafica disponibile via WEB implementata con tecnologie HTML5/CSS3 e con l'utilizzo di WebSocket.
Questa dashboard grafica comunicherà con delle API programmabili che effettueranno tutte le operazioni di gestione del prodotto implementando i requisiti esplicitati nei prossimi capitoli del presente documento.
\end{document}
