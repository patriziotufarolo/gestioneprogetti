\documentclass[../main.tex]{subfiles}
\begin{document}
\chapter{Casi di test di unità}
Lo unit testing viene normalmente eseguito dagli sviluppatori, e può essere occasionalmente glass box, ovvero essere esplicitamente basato sulla conoscenza dell'architettura e del funzionamento interno di un componente oltre che sulle sue funzionalità esternamente esposte.

Come le altre forme di testing, lo unit testing può variare da completamente "manuale" ad automatico.

Specialmente nel caso dello unit testing automatico, lo sviluppo dei test case (cioè delle singole procedure di test) può essere considerato parte integrante dell'attività di sviluppo (per esempio, nel caso dello sviluppo guidato da test).
\clearpage

\begin{tabularx}{\linewidth}{|c|X|X|X|c|}
    \toprule
    \textbf{Codice}      & \textbf{Classe}        & \textbf{Metodo}        & \textbf{Oracolo} & \textbf{Autom.} \\ 
    \midrule
    \endhead
    UT001       & WebService    & login         & Errore se il login con credenziali valide fallisce & Si \\ \hline
    UT002       & WebService    & login         & Errore se il login con credenziali errate va a buon fine & Si \\ \hline
    UT003       & WebService    & getEnvironments         & Errore se non c'è alcun ambiente configurato & Si \\ \hline
    UT005       & WebService    & createDevice         & Errore se fallisce la creazione di un device & Si \\ \hline
    UT006       & WebService    & createLink         & Errore se fallisce la creazione di un link & Si \\ \hline
    UT007       & WebService    & generateLink         & Errore se il link di invito non viene generato & Si \\ \hline
    UT008       & WebService    & deleteElements         & Errore se dopo l'eliminazione gli elementi esistono ancora & Si \\ \hline
    UT009       & WebService    & sendConfiguration         & Errore se una configurazione valida non viene validata & Si \\ \hline
    UT011       & WebService    & createSession         & Errore se la sessione non viene creata & Si \\ \hline
    UT012       & WebService    & getSession         & Errore se i dati di sessione non vengono restituiti & Si \\ \hline
    UT013       & WebService    & retrieveUsers         & Errore se il sistema non riesce ad ottenere la lista degli utenti & Si \\ \hline
    UT014       & WebService    & associate         & Errore se dopo l'esecuzione del metodo gli utenti non risultano associati alla sessione di lavoro & Si \\ \hline
    UT015       & WebService    & logout         & Errore se la sessione utente non viene cancellata dopo il logout & Si \\ \hline
    UT017       & WebUI         & do         & Errore se la richiesta POST associata non viene eseguita & Si\\ \hline
    UT018       & WebUI         & populate         & Errore se la pagina non viene popolata con i dati ottenuti dalla GET & Si \\ \hline
    UT019       & WebUI         & show         & Errore se una pagina richiesta non esiste o non può essere caricata & Si \\ \hline
    UT020       & TerminalProxy & openTerminal & Errore se la sessione terminale per un host non viene aperta & Si \\ \hline
    UT021       & TerminalProxy & attachTTYToSocket & Errore se la socket TTY viene chiusa inaspettatamente & Si \\ \hline
\end{tabularx}
\end{document}
