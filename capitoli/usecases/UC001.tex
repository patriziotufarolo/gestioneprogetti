\documentclass[../../main.tex]{subfiles}
\begin{document}
\subsection{Caso d’uso UC001: Login alla piattaforma }
\begin{tabularx}{150mm}{|l|X|}
    \hline
    ID                                  & \textbf{UC001}\\
    \hline
    Titolo                              & Login alla piattaforma \\
    \hline
    Descrizione breve                   & Un utente fa login alla piattaforma con le credenziali memorizzate nel database di autenticazione dell'organizzazione di appartenenza   \\
    \hline
    Attore primario                     Utente del sistema \\
    \hline
    Eventuali altri attori              & -   \\
    \hline
    Stakeholders e rispettivi interessi & L'utente del sistema necessita di utilizzare il sistema per condurre un'esercitazione o un esperimento. Nel caso in cui l'utente sia un supervisore di sessione, vuole poter creare una nuova sessione di lavoro ed associare alla stessa degli utenti   \\
    \hline
    Pre-condizioni                      & L'utente ha un username e una password per effettuare il login   \\
    \hline
    Post-condizioni                     & L'utente ha accesso al sistema e può condurre le azioni desiderate, in base ai privilegi ad esso garantiti   \\
    \hline
    Scenario principale                 & \begin {enumerate}
\item {L'utente visita la pagina web della piattaforma}
\item {L'utente clicca sul link login}
\item {L'utente inserisce le proprie credenziali}
\item {Il sistema effettua un check sulle credenziali e accetta l'accesso}
    \end{enumerate}
    \\
    \hline
    Scenari alternativi                 &
    \begin {enumerate}
\item{L'utente visita la pagina web della piattaforma}
\item{L'utente clicca sul link login}
\item{L'utente inserisce credenziali errate}
\item{Il sistema effettua un check sulle credenziali e nega l'accesso}
\item{L'utente inserisce credenziali corrette}
\item{Il sistema effettua un check sulle credenziali e accetta l'accesso}
    \end{enumerate}
    \\
    \hline
    Requisiti speciali                  &    \\
    \hline
    Eventuali punti aperti              &    Inserire possibilità di ripristinare la password sul database LDAP in caso di password dimenticata \\
    \hline
\end{tabularx}
\newpage
\end{document}
