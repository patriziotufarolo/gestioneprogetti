\documentclass[../../main.tex]{subfiles}
\begin{document}
\subsection{Caso d’uso UC006: Eliminazione di un elemento dal canvas}
\begin{tabularx}{150mm}{|l|X|}
    \hline
    ID                                  & \textbf{UC006}\\
    \hline
    Titolo                              & Eliminazione di un elemento dal canvas\\
    \hline
    Descrizione breve                   & L'utente può eliminare qualunque elemento del canvas, link o dispositivo che sia.   \\
    \hline
    Attore primario                     & Utente del sistema   \\
    \hline
    Eventuali altri attori              & -   \\
    \hline
    Stakeholders e rispettivi interessi & L'utente del sistema deve essere in grado di eliminare un elemento precedentemente creato e non più necessario all'esperimento   \\
    \hline
    Pre-condizioni                      & L'elemento da eliminare deve essere presente   \\
    \hline
    Post-condizioni                     & L'elemento selezionato è stato eliminato   \\
    \hline
    Scenario principale                 & 
    \begin{enumerate}
        \item L'utente fa click su un elemento del canvas con il tasto destro del mouse
        \item Il sistema propone un menù contestuale
        \item L'utente clicca sulla voce "elimina"
        \item L'elemento viene eliminato
    \end{enumerate}
    \\
    \hline
    Scenari alternativi                 &
    \begin{enumerate}
        \item L'utente fa click su un elemento del canvas con il tasto sinistro del mouse
        \item L'elemento viene selezionato
        \item L'utente preme il dasto DEL sulla tastiera
        \item L'elemento viene eliminato
    \end{enumerate}
    \begin{enumerate}
        \item L'utente esegue i punti 1,2,3 dei casi precedenti
        \item Il sistema va in errore
        \item Viene proposto all'utente di inviare opzionalmente un report sull'errore
    \end{enumerate} \\
    \hline
    Requisiti speciali                  &    L'utente deve essere informato che l'operazione di eliminazione è permanente e irreversibile \\
    \hline
    Eventuali punti aperti              &    Implementare tecniche di snapshotting per rendere l'operazione reversibile\\
    \hline
\end{tabularx}
\newpage
\end{document}
