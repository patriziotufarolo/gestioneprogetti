\documentclass[../../main.tex]{subfiles}
\begin{document}
\subsection{Caso d’uso UC008: Configurazione di un link }
\begin{tabularx}{150mm}{|l|X|}
    \hline
    ID                                  & \textbf{UC008}\\
    \hline
    Titolo                              & Configurazione di un link \\
    \hline
    Descrizione breve                   & L'utente configura le proprietà di un link. Alcune di queste possono essere la bandwidth, il bit error ratio, la latenza.   \\
    \hline
    Attore primario                     & Utente del sistema   \\
    \hline
    Eventuali altri attori              & -   \\
    \hline
    Stakeholders e rispettivi interessi & L'utente vuole poter simulare condizioni di carico sulla rete, malfunzionamenti della stessa, link con bandwidth ridotta rispetto a quella di default   \\
    \hline
    Pre-condizioni                      &  Il link deve esistere  \\
    \hline
    Post-condizioni                     &  Le proprietà del link ad esperimento avviato rispecchieranno quelle dichiarate  \\
    \hline
    Scenario principale                 &  
     \begin{enumerate}
        \item L'utente seleziona un link
        \item L'utente clicca sul tasto configura
        \item Il sistema propone una finestra di dialogo per la configurazione del link 
        \item L'utente inserisce la configurazione nella finestra di dialogo 
        \item L'utente applica la configurazione
    \end{enumerate}\\
    \hline
    Scenari alternativi                 &  
    \begin{enumerate}
        \item L'utente fa doppio click sul link
        \item Il sistema propone una finestra di dialogo e tutto procede come da punti 3, 4, 5 dello scenario principale
    \end{enumerate}
    \begin{enumerate}
        \item L'utente esegue i punti 1, 2, 3, 4 dello scenario principale
        \item L'utente fa click sul pulsante "Annulla"
        \item La configurazione non viene applicata
    \end{enumerate} \\
    \hline
    Requisiti speciali                  & -   \\
    \hline
    Eventuali punti aperti              & -   \\
    \hline
\end{tabularx}
\newpage
\end{document}
