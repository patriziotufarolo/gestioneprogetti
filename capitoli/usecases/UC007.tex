\documentclass[../../main.tex]{subfiles}
\begin{document}
\subsection{Caso d’uso UC007: Configurazione di un dispositivo  }
\begin{tabularx}{150mm}{|l|X|}
    \hline
    ID                                  & \textbf{UC007}\\
    \hline
    Titolo                              & Configurazione di un dispositivo \\
    \hline
    Descrizione breve                   & L'utente configura il dispositivo in fase di pre-deployment. Se il dispositivo è di livello 2 ISO/OSI (es. switch, bridge) i parametri possono riguardare la politica di scheduling e queueing dei pacchetti, le porte del dispositivo, i VLAN tag, porte di trunk, politiche SDN etc.   
    Se il dispositivo è di livello 3 ISO/OSI, i parametri possono riguardare la configurazione IPv4 e IPv6 (IP/SUBNET/GATEWAY), routing, configurazioni particolari sul kernel (es. forwarding dei pacchetti, firewall) \\
    \hline
    Attore primario                     & Utente del sistema   \\
    \hline
    Eventuali altri attori              & -   \\
    \hline
    Stakeholders e rispettivi interessi & L'utente del sistema vuole configurare in modo idoneo ai suoi scopi il dispositivo in oggetto   \\
    \hline
    Pre-condizioni                      & Il dispositivo deve essere presente sulla topologia.\\& L'esperimento non deve essere stato avviato    \\
    \hline
    Post-condizioni                     & Il dispositivo è configurato in base alle esigenze dell'utente   \\
    \hline
    Scenario principale                 &
    \begin{enumerate}
        \item L'utente seleziona un dispositivo
        \item L'utente clicca sul tasto configura
        \item Il sistema propone una finestra di dialogo
        \item L'utente inserisce la configurazione nella finestra di dialogo in formato \textit{JSON}
        \item L'utente applica la configurazione
    \end{enumerate}
    \\
    \hline
    Scenari alternativi                 &    
    \begin{enumerate}
        \item L'utente fa doppio click sul dispositivo
        \item Il sistema propone una finestra di dialogo e tutto procede come da punti 3, 4, 5 dello scenario principale
    \end{enumerate}
    \begin{enumerate}
        \item L'utente esegue i punti 1, 2, 3, 4 dello scenario principale
        \item L'utente fa click sul pulsante "Annulla"
        \item La configurazione non viene applicata
    \end{enumerate}
    \\
    \hline
    Requisiti speciali                  & -   \\
    \hline
    Eventuali punti aperti              &    Sviluppare UI ad-hoc per la gestione delle configurazioni eliminando l'effort del formato \textit{JSON}\\
    \hline
\end{tabularx}
\newpage
\end{document}
