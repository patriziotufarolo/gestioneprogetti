\documentclass[../../main.tex]{subfiles}
\begin{document}
\subsubsection{Caso d’uso UC003: Selezione della sessione di lavoro }
\begin{tabularx}{150mm}{|l|X|}
    \hline
    ID                                  & \textbf{UC003}\\
    \hline
    Titolo                              & Selezione della sessione di lavoro \\
    \hline
    Descrizione breve                   & L'utente dopo aver scelto l'ambiente su cui vuole lavorare, ottiene una lista di tutte le sessioni di lavoro attive per quell'ambiente. Può quindi scegliere la sessione di lavoro relativa.   \\
    \hline
    Attore primario                     & Utente   \\
    \hline
    Eventuali altri attori              & Supervisore di sessione   \\
    \hline
    Stakeholders e rispettivi interessi & L'utente vuole poter accedere a una sessione per poter lavorare   \\
    \hline
    Pre-condizioni                      & L'utente ha scelto un ambiente ed è associato a una sessione di lavoro nell'ambiente scelto   \\
    \hline
    Post-condizioni                     & L'utente può vedere il proprio canvas di lavoro   \\
    \hline
    Scenario principale                 & \begin{enumerate}
        \item L'utente esegue UC001 e UC002
        \item Il sistema propone le sessioni di lavoro abilitate per quell'utente
        \item L'utente sceglie la sessione di lavoro tra quelle proposte
    \end{enumerate} \\
    \hline
    Scenari alternativi &
    \begin{enumerate}
        \item L'utente esegue UC001 e UC002
        \item L'utente non è associato a sessioni di lavoro
        \item Viene mostrato un messaggio informativo
    \end{enumerate}
        \\
    \hline
    Requisiti speciali                  & -   \\
    \hline
    Eventuali punti aperti              & -   \\
    \hline
\end{tabularx}
\newpage
\end{document}
