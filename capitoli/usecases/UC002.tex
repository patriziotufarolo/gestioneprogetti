\documentclass[../../main.tex]{subfiles}
\begin{document}
\subsection{Caso d’uso UC002: Selezione dell'ambiente }
\begin{tabularx}{150mm}{|l|X|}
    \hline
    ID                                  & \textbf{UC002}\\
    \hline
    Titolo                              & Selezione dell'ambiente \\
    \hline
    Descrizione breve                   & L'utente, dopo aver effettuato il login, vuole selezionare un ambiente di lavoro tra quelli disponibili. Ad esempio, per condurre un esperimento in "Reti di calcolatori", seleziona l'ambiente "Reti".   \\
    \hline
    Attore primario                     & Utente del sistema
    \\
    \hline
    Eventuali altri attori              & Amministratore di sistema \textit{(Attore esterno)}   \\
    \hline
    Stakeholders e rispettivi interessi & L'utente vuole effettuare una sessione di lavoro in un determinato ambiente   \\
    \hline
    Pre-condizioni                      & Deve essere stato installato e configurato sulla piattaforma almeno un ambiente   \\
    \hline
    Post-condizioni                     & All'utente vengono mostrate le sessioni di lavoro disponibili per l'ambiente da egli scelto  \\
    \hline
    Scenario principale                 & \parbox[t][2.4cm]{8cm}{\begin{enumerate}
        \item L'utente effettua UC001
        \item L'utente seleziona l'ambiente tra quelli proposti dal sistema
\end{enumerate}}
            \\
    \hline
    Scenari alternativi                 & \parbox[t][2.4cm]{8cm}{\begin {enumerate}
        \item L'utente effettua UC001
        \item Non è stato configurato nessun ambiente nella piattaforma, viene mostrato un messaggio informativo e la piattaforma non è utilizzabile
\end{enumerate}}
            \\
    \hline
    Requisiti speciali                  &    \\
    \hline
    Eventuali punti aperti              &    Introduzione di un form per segnalare il problema all'amministratore di sistema \\
    \hline
\end{tabularx}
\newpage
\end{document}
