\documentclass[../../main.tex]{subfiles}
\begin{document}
\subsection{Caso d’uso UC005: Aggiunta di un link tra due dispositivi }
\begin{tabularx}{150mm}{|l|X|}
    \hline
    ID                                  & \textbf{UC005}\\
    \hline
    Titolo                              & Aggiunta di un link tra due dispositivi \\
    \hline
    Descrizione breve                   & L'utente traccia un collegamento tra i due dispositivi   \\
    \hline
    Attore primario                     & Utente del sistema   \\
    \hline
    Eventuali altri attori              & -  \\
    \hline
    Stakeholders e rispettivi interessi & L'utente vuole mettere in comunicazione due dispositivi   \\
    \hline
    Pre-condizioni                      & L'utente ha creato due dispositivi   \\
    \hline
    Post-condizioni                     & I due dispositivi (peer del link) sono collegati fra loro e, ad esperimento avviato, potranno comunicare   \\
    \hline
    Scenario principale                 &
    \begin {enumerate}
\item L'utente crea due dispositivi come da UC004
\item L'utente clicca sull'icona "Aggiungi link"
\item L'utente traccia una linea sul canvas unendo i nodi coinvolti sul link
\item Il link viene creato e viene visualizzato nel canvas
    \end {enumerate}
    \\
    \hline
    Scenari alternativi                 &    
    \begin {enumerate}
\item L'utente crea due dispositivi come da UC004
\item L'utente clicca sull'icona "Aggiungi link" con il tasto destro del mouse
\item Viene proposta una finestra di dialogo che consente la selezione dei nodi coinvolti nel link 
\item L'utente seleziona i nodi e preme "Ok" o il tasto invio
\item Il link viene creato e viene visualizzato nel canvas
    \end {enumerate} \\

    \hline
    Requisiti speciali                  &    -\\
    \hline
    Eventuali punti aperti              &    Specificare eventuali proprietà del link al momento della creazione dello stesso\\
    \hline
\end{tabularx}
\newpage
\end{document}
